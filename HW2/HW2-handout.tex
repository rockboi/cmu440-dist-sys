\documentclass{article}
\usepackage[utf8]{inputenc}
\usepackage[margin=1.2in]{geometry}
\usepackage{graphicx}

\def\answer#1{
    \vspace*{0.5em}\\
    \noindent\fbox{%
    \parbox{\textwidth}{%
		#1
    }%
}
}

\newcommand{\mycoursenum}{15-440/15-640}
\newcommand{\myhwnum}{2}
\newcommand{\myname}{}   % put your name here
\newcommand{\myandrew}{}  % put your andrewID here

\newif\ifprint
\printfalse  % change to printfalse after adding your name and andrewID

\def\changemargin#1#2{\list{}{\rightmargin#2\leftmargin#1}\item[]}
\let\endchangemargin=\endlist 

\begin{document}
\medskip
\thispagestyle{plain}
\begin{center}
{\Large \mycoursenum: Homework \myhwnum}\\
Due: October 13, 2017 10:30am (\large\textbf{NO LATE DAY}) \\
\medskip
\begin{tabular}{| l p{3in}|}
\hline
Name: \myname& \\
& \\ \hline
Andrew ID: \myandrew & \\
&\\
\hline
\end{tabular}
\end{center}



%%%%%%%%%%%%%%%%%%%%%%%%%%%%%%%%%%%%%%%%%%%%%%%%%%%%%%%%%%%%%%%%%%%%%%%%%%%%%%%%%%

\section{Concurrency Control}
\begin{enumerate}
\item Is 2-phase commit blocking or non-blocking? What about 3-phase commit? Elaborate on your answers by explaining the situations where a transaction blocks or why it does not block.
    % uncomment this to add answer
    %\answer{}
    
\item Consider the following program:
\begin{verbatim}
// DB is a key/value database
acquireWriteLock(key1)
acquireWriteLock(key2)
x = DB[key1]
DB[key2] = x
releaseLock(key2)
acquireReadLock(key3)
y = DB[key3]
DB[key1] = y
releaseLock(key3)
releaseLock(key1)
\end{verbatim}
\begin{enumerate}
\item Which ACID property(s) does this program violate?
\item How would strong strict 2-phase locking fix it?
\end{enumerate}
\end{enumerate}

%%%%%%%%%%%%%%%%%%%%%%%%%%%%%%%%%%%%%%%%%%%%%%%%%%%%%%%%%%%%%%%%%%%%%%%%%%%%%%%%%%

\section{Distributed Mutual Exclusion}
\begin{enumerate}
\item In class we discussed enforcing mutual exclusion, among other ways, via a central server, majority voting, and token ring. 
    \begin{enumerate}
    \item How many messages are required per request under heavy contention? What are the messages used for? Write your answer for each algorithm.
    % uncomment this to add answer
    %\answer{}
    \item List at least one disadvantage of each algorithm.
    % uncomment this to add answer
    %\answer{}
    \item Which of these systems is most robust to failure? Why?
    % uncomment this to add answer
    %\answer{}
    \end{enumerate}
\item Consider three processes. The system has totally ordered clocks by breaking ties by process ID. It uses the Ricard $\&$ Agrawala algorithm. The timestamp for each process of id $i$ is $T(p) = 10*L(p) + i$, where $L(p)$ is a regular Lamport clock. 

Each message takes 2 “real-time” steps to get delivered. Critical section takes 2 real-time steps. Fill in the table with the messages that are being broadcast, sent, or received between the processes until all nodes have executed their critical sections. Write “execute critical section” as the action for a node when it enters its critical section. The first three rows have been filled in for you, and the fourth row has been started. Assume that if a process receives messages from the other two processes at the same time, the message that comes from the lower process ID will be received first.  


Action Types: Broadcast (B), Receive (R), Send (S),  Execute Critical Section (ExCS)
Initial timestamps: P1 $\rightarrow$ 111, P2 $\rightarrow$ 212 and P3 $\rightarrow$ 103

\begin{changemargin}{-0.5in}{0.5in} 
\begin{tabular}{|c|c|c|c|c|c|c|c|}
\hline
Real Time & Process & Lamport Time & Action(to/from) & Contents & Q at P1 & Q at P2 & Q at P3 \\
\hline
1 & 1 & 111 & B & (request 111) & 111 & & 123 \\
 & 3 & 123 & B & (request 123) & & & \\
\hline
 2 & 2 & 222 & B & (request 222) & 121 & 222 & 123 \\
  & & & & & & & \\
  & & & & & & & \\
\hline
  3 & 1 & 131 & R from 3 & (request 123) & 111 & 111 & 111 \\
  & 2 & 232 & R from 1 & (request 111) & 123 & 123 & 123 \\
  & 2 & 242 & R from 3 & (request 123) & & 222 & \\
  & 3 & 133 & R from 1 & (request 111) & & & \\
\hline
  4 & 1 & 231 & R from 2 & (request 222) & 111 & 222 & 123 \\
  & 2 & 252 & S to 1 & (reply 111) & 123 &  & 222 \\
  & 2 & 262 & S to 3 & (reply 123) & 222 & & \\
  & 3 & 233 & R from 2 & (request 222) & &  &  \\
  & 3 & 243 & S to 1 & (reply 111) & & & \\
\hline
  & & & & & & &\\
  & & & & & & &\\
  & & & & & & &\\
\hline
  & & & & & & &\\
  & & & & & & &\\
  & & & & & & &\\
\hline
 & & & & & & &\\
  & & & & & & &\\
  & & & & & & &\\
\hline
  & & & & & & &\\
  & & & & & & &\\
  & & & & & & &\\
\hline
  & & & & & & &\\
  & & & & & & &\\
  & & & & & & &\\
\hline
  & & & & & & &\\
  & & & & & & &\\
  & & & & & & &\\
\hline
  & & & & & & &\\
  & & & & & & &\\
  & & & & & & &\\
\hline
  & & & & & & &\\
  & & & & & & &\\
  & & & & & & &\\
\hline
  & & & & & & &\\
  & & & & & & &\\
  & & & & & & &\\
\hline
  & & & & & & &\\
  & & & & & & &\\
  & & & & & & &\\
\hline
  & & & & & & &\\
  & & & & & & &\\
  & & & & & & &\\
  \hline
  & & & & & & &\\
  & & & & & & &\\
  & & & & & & &\\
  \hline
  & & & & & & &\\
  & & & & & & &\\
  & & & & & & &\\
\hline
\end{tabular}
\end{changemargin}
\end{enumerate}
%%%%%%%%%%%%%%%%%%%%%%%%%%%%%%%%%%%%%%%%%%%%%%%%%%%%%%%%%%%%%%%%%%%%%%%%%%%%%%%%%%

\section{Logging and Crash Recovery}
\begin{enumerate}
\item The ARIES logging and crash recovery design we talked about can also recover
from a crash during the recovery phase for a previous crash. How is this achieved?
\item List one advantage and one disadvantage of checkpointing in a log based recovery system.
\end{enumerate}

%%%%%%%%%%%%%%%%%%%%%%%%%%%%%%%%%%%%%%%%%%%%%%%%%%%%%%%%%%%%%%%%%%%%%%%%%%%%%%%%%%

\section{Distributed Replication/Paxos}
\begin{enumerate}
\item Justify the need for the prepare phase in the Paxos algorithm.
\item You have set up a fault-tolerant banking service for the PNC bank (Paxos National Corporation bank).  Based upon an examination of other systems, you've decided that the best way to do so is to use the Paxos algorithm to replicate log entries across three servers, and let one of your employees handle the issue of recovering from a failure using the log.

The state on the replicas consists of a list of all bank account
mutation operations that have been made,
each with a unique request ID to prevent retransmitted requests,
listed in the order they were committed.

Assume that the replicas execute Paxos for every
operation.\footnote{In practice, most systems use Paxos to elect a
  primary and let it have a lease on the operations for a while, but
  that adds complexity to the homework problem.}  Each value that the
servers agree on looks like ``account action 555 transfers \$1,000,000 from
Yuvraj A. to Srini S.''.  When a server receives a request, it looks at its state to find the next unused action number, and uses Paxos to propose that value for the number to use.

The three servers are S1, S2, and S3.

\emph{At the same time}:
\begin{itemize}
  \item S1 receives a request to withdraw \$500 from Yuvraj.
  \begin{itemize}
    \item S1 picks proposal number 501 (the $n$ in Paxos)
  \end{itemize}
  \item S2 receives a request to transfer \$500 from Yuvraj to Srini.
  \begin{itemize}
    \item S2 picks proposal number 502
  \end{itemize}
\end{itemize}

Both servers look at their lists of applied account actions and decide
that the next action number is 15. So both start executing Paxos as a
leader for action 15.\\

The first few messages are given below:
\begin{verbatim}
S1 -> S1 PREPARE(501) 
S1 -> S1 RESPONSE(nil, nil)

S1 -> S2 PREPARE(501) 
S2 -> S1 RESPONSE(nil, nil)

S1 -> S3 PREPARE(501) 
S3 -> S1 RESPONSE(nil, nil)
\end{verbatim}

For each of the following scenarios, give a sequence of events that could
lead to it.
\begin{enumerate}
\item The servers agree on the withdrawal as entry 15.
\item The servers agree on the transfer as entry 15.
\end{enumerate}
\end{enumerate}

%%%%%%%%%%%%%%%%%%%%%%%%%%%%%%%%%%%%%%%%%%%%%%%%%%%%%%%%%%%%%%%%%%%%%%%%%%%%%%%%%%

\section{Fault tolerance and RAID}
A friend of yours bought 25 used hard drives from a sale. Each drive has the
following characteristics:
\begin{itemize}
\item Capacity: 500 GB
\item Sequential speed: 60 MB/s
\item Random IOPS: 30 IOPS
\item MTTF: 1 year
\end{itemize}
Your friend wants a lot of speed and a lot of storage, and decides to 
build a RAID-0 array with the disks.
\begin{enumerate}
\item Is this a good idea? Discuss this w.r.t. the effective capacity,
sequential speed, random IOPS, the MTTF and data durability.
\item Another friend suggested that a RAID-5 array with 25 disks would be a better idea
than RAID-0. Why? Compare all the characteristics of the RAID-5 arrangement with the RAID-0
arrangement.
\item Your friend was a little disappointed on hearing that the system would fail
very soon and new disks would be needed to replace the failing ones. Your friend agrees to
compromise a little on storage space and speed. What would be a good design so that
no additional disks need to be purchased for about a year?
\end{enumerate}

%%%%%%%%%%%%%%%%%%%%%%%%%%%%%%%%%%%%%%%%%%%%%%%%%%%%%%%%%%%%%%%%%%%%%%%%%%%%%%%%%%
\clearpage

\end{document}

